\documentclass{amsart}

\usepackage[english]{babel}
\usepackage[utf8]{inputenc}
\usepackage{graphicx}
\usepackage{mathtools}
\usepackage{amsthm}
\usepackage{thmtools,thm-restate}
\usepackage{amsfonts}
\usepackage{hyperref}
\usepackage[singlelinecheck=false]{caption}
\usepackage[backend=biber,url=true,doi=true,eprint=false,style=alphabetic]{biblatex}
\usepackage{enumitem}
\usepackage[justification=centering]{caption}
\usepackage{indentfirst}
\usepackage{algorithm}
\usepackage{algpseudocode}
\usepackage{listings}
\usepackage[x11names, rgb]{xcolor}
\usepackage{tikz}
\usepackage{hyperref}
\usetikzlibrary{snakes,arrows,shapes}

\addbibresource{references.bib}

\makeatletter
\def\subsection{\@startsection{subsection}{3}%
  \z@{.5\linespacing\@plus.7\linespacing}{.1\linespacing}%
  {\normalfont}}
\makeatother

\makeatletter
\patchcmd{\@setauthors}{\MakeUppercase}{}{}{}
\makeatother

\DeclareMathOperator*{\argmin}{arg\,min}
\DeclareMathOperator*{\argmax}{arg\,max}
\DeclareMathOperator*{\Val}{\text{Val}}
\DeclareMathOperator*{\Ch}{\text{Ch}}
\DeclareMathOperator*{\Pa}{\text{Pa}}
\DeclareMathOperator*{\Sc}{\text{Sc}}
\newcommand{\ov}{\overline}

\newcommand\defeq{\mathrel{\overset{\makebox[0pt]{\mbox{\normalfont\tiny\sffamily def}}}{=}}}

\newcommand{\algorithmautorefname}{Algorithm}
\algrenewcommand\algorithmicrequire{\textbf{Input}}
\algrenewcommand\algorithmicensure{\textbf{Output}}

\captionsetup[table]{labelsep=space}

\theoremstyle{plain}

\newcounter{dummy-def}\numberwithin{dummy-def}{section}
\newtheorem{definition}[dummy-def]{Definition}
\newcounter{dummy-thm}\numberwithin{dummy-thm}{section}
\newtheorem{theorem}[dummy-thm]{Theorem}
\newcounter{dummy-prop}\numberwithin{dummy-prop}{section}
\newtheorem{proposition}[dummy-prop]{Proposition}
\newcounter{dummy-corollary}\numberwithin{dummy-corollary}{section}
\newtheorem{corollary}[dummy-corollary]{Corollary}
\newcounter{dummy-lemma}\numberwithin{dummy-lemma}{section}
\newtheorem{lemma}[dummy-lemma]{Lemma}
\newcounter{dummy-ex}\numberwithin{dummy-ex}{section}
\newtheorem{exercise}[dummy-ex]{Exercise}
\newcounter{dummy-eg}\numberwithin{dummy-eg}{section}
\newtheorem{example}[dummy-eg]{Example}

\numberwithin{equation}{section}

\newcommand{\set}[1]{\mathbf{#1}}
\newcommand{\pr}{\mathbb{P}}
\renewcommand{\implies}{\Rightarrow}

\newcommand{\bigo}{\mathcal{O}}

\setlength{\parskip}{1em}

\lstset{frameround=fttt,
	numbers=left,
	breaklines=true,
	keywordstyle=\bfseries,
	basicstyle=\ttfamily,
}

\newcommand{\code}[1]{\lstinline[mathescape=true]{#1}}
\newcommand{\mcode}[1]{\lstinline[mathescape]!#1!}


\title{%
  \noindent\rule{13cm}{1.0pt}\\
  \vspace{0.2cm}
  Analysis on an Implementation of the Gens-Domingos Sum-Product Network Structural Learning
  Schema
  \noindent\rule{13cm}{0.8pt}
}
\xdef\shorttitle{Analysis on the GD Schema}
\author[]{\normalsize\textbf{Renato Lui Geh}\\\small Computer Science\\Institute of Mathematics
  and Statistics\\University of São Paulo\\\texttt{renatolg@ime.usp.br}}

\begin{document}

\begin{abstract}
  Sum-Product Networks (SPNs) are a class of deep probabilistic graphical models. Inference in them
  is linear in the number of edges of the graph. Furthermore, exact inference is achieved, in a
  valid SPN, by running through its edges twice at most, making exact inference linear. The
  Gens-Domingos SPN Schema is an algorithm for structural learning on such models. In this paper we
  present an implementation of such schema, analyzing its complexity, discoursing implementational
  and theoretical details, and finally presenting results and experiments achieved with this
  implementation.

  \smallskip
  \smallskip
  \smallskip
  \textbf{Keywords}
  \smallskip
  \texttt{cluster analysis; data mining; probabilistic graphical models; tractable models; machine
  learning; deep learning}
  \vspace*{-3.5em}
\end{abstract}

\maketitle

\section{Introduction}

A Sum-Product Network (SPN) is a probabilistic graphical model that represents a tractable
distribution of probability. If an SPN is valid, then we can perform exact inference in time linear
to the graph's edges. Its syntax is different to other conventional models (read bayesian and
markov networks) in the sense that its graph does not explicitly model events and (in) dependencies
between variables. That is, whilst variables in a bayesian network are represented as nodes in the
graph, with each edge connecting two nodes asserting a dependency relationship between the
connected variables, a node in an SPN may not necessarily represent a variable or event, neither an
edge connecting two nodes represent dependence. In this sense, SPNs can be seen as a type of
probabilistic Artificial Neural Network (ANN). However, whilst neural networks represent a
function, SPNs model a tractable probability distribution. Furthermore, SPNs are distinct from
standard neural networks seeing that, whereas ANNs have only one type of neuron with an activation
function mapping to values in $[0,1]$, SPNs have two kind of neurons, which we will see in the next
sections. Still, SPNs retain certain important characteristics from ANNs as we will discuss later,
with mainly its deep structure properties~\cite{shallow-vs-deep} as the most interesting feature.

The Gens-Domingos Schema~\cite{gens-domingos}, or \code{LearnGD} as we will reference it throughout
this paper, is an SPN structural learning algorithm proposed by Robert Gens and Pedro Domingos.
Gens and Domingos call it a schema because it only provides a template of what the algorithm should
be like. We will discuss \code{LearnGD} in details in the next section. This paper documents a
particular implementation of the GD schema. Other implementations may have different results.

In this document, we show how we implemented the \code{LearnGD} algorithm. We analyse the
complexity of each algorithm component in detail, later referring to such analyses when drawing
conclusions on the overall complexity of the algorithm. As we have mentioned before, since the
\code{LearnGD} schema depends heavily on implementation, the complexity we achieve in this
particular case may differ from other implementations. After each analysis, we then look at the
algorithm as whole, drawing conclusions on time and memory usage, as well as implementation details
that could potentially decrease the algorithm runtime. We also comment on how to implement better
concurrency then how it is currently coded in our implementation. We then show some results on
experiments made on image classification and image completion.

\section{Sum-Product Networks}

In this section we will define SPNs differently from other articles~\cite{gens-domingos,
poon-domingos, clustering} as the original more convoluted definition is of little use for the
\code{LearnGD} algorithm. Our definition is almost identical to the original \code{LearnGD} article
\cite{gens-domingos}, with the exception that we assume that an SPN is already normalized. This
fact changes nothing, since Peharz \textit{et al} recently proved that normalized SPNs have as much
representability power as unnormalized SPNs~\cite{theoretical-spn}. Before we enunciate the formal
definition of an SPN, we will give an informal, vague definition of an SPN in order to explain what
completeness, consistency, validity and decomposability --- which are an important set of
definitions --- of an SPN mean.

A sum-product network represents a tractable probability distribution through a DAG\@. Such digraph
must always be weakly connected. A node can either be a leaf, a sum, or a product node. The scope
of a node is the set of all variables present in all its descendants. Leaf nodes are tractable
probability distributions and their scope is the scope of its distribution, sum nodes represent the
summing out of the variables in its scope and product nodes act as feature hierarchy. An edge that
has its origin from a sum node has a non-negative weight. We refer to a sub-SPN $S$ rooted at node
$i$ as $S(i)$, while the SPN rooted at its root is denoted as $S(\cdot)$ or simply $S$. The scope
of a node will be denoted as $\Sc(i)$, where $i$ is a node. The set of children of a node will be
denoted as $\Ch(i)$. Similarly, $\Pa(i)$ is the set of parents of node $i$.

\begin{definition}[Normalized]~\\
  Let $S$ be an SPN and $\Sigma(S)$ be the set of all sum nodes of $S$. $S$ is normalized iff, for
  all $\sigma \in \Sigma(S)$, $\sum_{c\in Ch(\sigma)} w_{\sigma c} = 1$ and $0 \leq w_{\sigma c}
  \leq 1$, where $w_{\sigma c}$ is the weight from edge $\sigma \to c$.
\end{definition}

\begin{definition}[Completeness]~\\
  Let $S$ be an SPN and $\Sigma(S)$ be the set of all sum nodes of $S$. $S$ is complete iff, for
  all $\sigma \in \Sigma(S)$, $\Sc(i)=\Sc(j), i\neq j; \forall i,j\in \Ch(\sigma)$.
\end{definition}

\begin{definition}[Consistency]~\\
  Let $S$ be an SPN, $\Pi(S)$ be the set of all product nodes of $S$ and $X$ a variable in
  $\Sc(S)$. $S$ is consistent iff $X$ takes the same value for all elements in $\Pi(S)$ that
  contain $X$.
\end{definition}

\begin{definition}[Validity]~\\
  An SPN $S$ is valid iff it always computes the correct probability of evidence $S$ represents.
\end{definition}

\begin{theorem}\label{thm:validity}
  An SPN $S$ is valid if it is both complete and consistent.
\end{theorem}

Validity guarantees that the SPN will compute not only the correct probability of evidence, but
also in time linear to its graph's edges. Therefore, it is preferable to learn valid SPNs. Notice
that~\autoref{thm:validity} is not restricted by completeness and consistency. In fact, incomplete
and/or inconsistent SPNs can compute the probability of evidence correctly, but consistency and
completeness guarantee that all sub-SPNs are also valid.

\begin{definition}[Decomposability]~\\
  Let $S$ be an SPN and $\Pi(S)$ be the set of all product nodes in $S$. $S$ is decomposable iff,
  for all $\pi \in \Pi(S)$, $\Sc(i)\cap \Sc(j)=\emptyset, i\neq j; \forall i,j\in \Ch(\pi)$.
\end{definition}

It is clear that decomposability implies consistency, therefore if an SPN is both complete and
decomposable, than it is also valid. We choose to work with decomposability because it is easier to
learn decomposable SPNs then it is to learn consistent ones. We do not lose representation power
because a complete and consistent SPN can be transformed into a complete and decomposable SPN in no
more than a polynomial number of edge and node additions~\cite{theoretical-spn}. We can now
formally define an SPN\@.

\begin{definition}[Sum-product network]~\\
  A sum-product network (SPN) is a weakly connected DAG that can be recursively defined as
  following.

  An SPN\@:
  \begin{enumerate}
    \item with a single node is a univariate tractable probability distribution (\textbf{leaf});
    \item is a normalized weighted sum of SPNs of same scope (\textbf{sum});
    \item is a product of SPNs with disjoint scopes (\textbf{product}).
  \end{enumerate}
  The value of an SPN is defined by its type. Let $\lambda$, $\sigma$ and $\pi$ be a leaf, sum and
  product respectively. The values of such SPNs are given by $\lambda(\mathbf{x})$,
  $\sigma(\mathbf{x})$ and $\pi(\mathbf{x})$, where $\mathbf{x}$ is a certain evidence
  instantiation.
  \begin{description}
    \item[Leaf] $\lambda(\mathbf{x})$ is the value of the probability distribution at point
      $\mathbf{x}$.
    \item[Product] $\pi(\mathbf{x}) = \prod_{c \in \Ch(\pi)} c(\mathbf{x})$.
    \item[Sum] $\sigma(\mathbf{x}) = \sum_{c \in \Ch(\sigma)} w_{\sigma c} c(\mathbf{x})$, with
      $\sum_{c \in \Ch(\sigma)} w_{\sigma c} = 1$ and $0 \leq w_{\sigma c} \leq 1$.
  \end{description}
\end{definition}

Note that this definition assumes an SPN to be complete, decomposable and normalized. Other
definitions in literature may differ from ours, but as we have mentioned before, for our
implementation, this definition is convenient for us. Another observation worthy of notice is the
value of $\lambda(\mathbf{x})$. Although here we consider $\mathbf{x}$ to be a multivariate
instantiation (i.e.\ a set of --- potentially multiple --- variable valuations), we had initially
defined a leaf to be a univariate distribution. Although it is possible to attribute leaves as
multivariate probability distributions~\cite{id-spn}, for our definition we have chosen to keep a
leaf's scope a unit set. Therefore, in the case of a leaf's value, $\mathbf{x}$ is a singleton
(univariate) variable instantiation.

\section{The \code{LearnGD} Schema}

The \code{LearnGD} schema was proposed by Robert Gens and Pedro Domingos on \textit{Learning the
Structure of Sum-Product Networks}~\cite{gens-domingos}. In this section we will outline the schema
in pseudo-code and analyse a few properties derived from the algorithm.

\begin{algorithm}[H]
  \caption{\code{LearnGD}}\label{alg:learngd}
  \begin{algorithmic}[1]
    \Require\,Set $D$ of instances (data)
    \Require\,Set $V$ of variables (scope)
    \Ensure\,An SPN representing a probability distribution given by $D$ and $V$
    \If{$|\set{V}|=1$} \Comment{univariate data sample}
      \State\,\textbf{return} univariate distribution estimated from $T[V]$ (data of $V$)
    \EndIf%
    \State\,Take $V$ and find mutually independent subsets $V_i$ of variables
    \If{possible to partition} \Comment{i.e.\ we have found independent subsets}
      \State\,\textbf{return} $\prod_i$ \mcode{LearnGD ($D$, $V_i$)}
    \Else\Comment{we cannot say there is independence}
      \State\,Take $D$ and find $D_j$ subsets of similar instances
      \If{possible to partition}
        \State\,\textbf{return} $\sum_i \frac{|D_j|}{|D|} \cdot$ \mcode{LearnGD ($D_j$, $V$)}
      \Else\Comment{i.e.\ data is one big cluster}
        \State\,\textbf{return} fully factorized distribution.
      \EndIf%
    \EndIf%
  \end{algorithmic}
\end{algorithm}

Let us now, for a moment, suppose that SPNs are not necessarily complete, decomposable and
normalized. We shall prove a few results derived from SPNs generated by~\autoref{alg:learngd}.

\begin{lemma}
  An SPN $S$ generated by \code{LearnGD} is complete, decomposable and normalized.
\end{lemma}
\begin{proof}
  Lines 4--6 show that the scope of each child in a product node of $S$ is a partition of the scope
  of their parent. Therefore, children have pairwise disjoint scopes on line 6, which proves
  decomposability for this part of the algorithm. In lines 8--10, since we are clustering similar
  instances, $D$ is being partitioned but we are not changing $V$ in any way. In fact, line 10
  shows that we pass $V$ to all other children. That is, all children of sum nodes have the same
  scope as their parent, which proves completeness.  Let $D_1,\ldots,D_n$ be the subsets of similar
  instances. By the definition of clustering, $D_1\cup\ldots\cup D_n=D$ and $D_i\cap D_j=
  \emptyset$, $i\neq j$, $1\leq i,j\leq n$. Thus it follows that $\sum_{i=1}^n \frac{|D_i|}{|D|}=1$
  and thus line 10 always creates complete and normalized sum nodes. Line 12 is a special case
  where, if we have discovered that $D$ is one big data cluster, we shall create a product node
  $\pi$ in which all children of $\pi$ are leaves and
  \begin{equation*}
    \bigcup_{\lambda\in\Ch(\pi)} \Sc(\lambda) = \Sc(\pi).
  \end{equation*}
  In other words, we fully factorize our product node into leaves. In this case, it is obvious that
  this product node is decomposable.
\end{proof}

\code{LearnGD} can be divided into three parts:

\begin{enumerate}
  \item Is the data univariate? If it is, return a leaf.
  \item Are partitions of the data independent? If they are, return a product node whose children
    are the independent partitions.
  \item Are partitions of the data similar? If they are, return a sum node whose children are the
    partition clusters.
  \item In case all else fails, we have a fully factorized distribution.
\end{enumerate}

Going back to our definition of an SPN, we can now take a more intuitive approach and make the
following observations:

\begin{enumerate}
  \item A leaf is nothing but a local/partitioned/sample distribution of a probability distribution
    given by a single variable.
  \item A product node determines independence between variables.
  \item A sum node is a clustering of similar data values (i.e.\ instances that are ``alike'').
\end{enumerate}

This gives more semantic value to SPNs, whilst still retaining its expressivity. Following this
approach, one can easily notice that each ``layer'' corresponds to a recursive call in
\code{LearnGD}. In fact, each recursive call constructs a hidden layer that tries to partition the
SPN even further. This gives SPNs a deep architecture that resembles deep models in that the deeper
the model, the more representation power it has~\cite{shallow-vs-deep}.

Let us now observe the scope of each type of node. A leaf is the trivial case, since it has a
single variable in its scope by definition. Each layer above it can have either sum or product
nodes. Let us now look at decomposability, that is: if a variable $X$ appears in a child of a
product node $\pi$, then $X$ cannot appear in another child of $\pi$. This gives us the following
result:

\begin{proposition}
  Let $S$ be an SPN generated by \code{LearnGD}, and let $\Lambda(S)$ be the set of all leaves of
  $S$. Then, $\forall \lambda \in \Lambda(S)$, we have that, $\forall p \in \Pa(\lambda)$, $p$ is a
  product node.
\end{proposition}
\begin{proof}
  Our proof is by contradiction. Let us assume that $\exists p \in \Pa(\lambda)$ such that $p$ is
  a sum node and $\exists c^* \in \Ch(p)$ a leaf. From our assumption that $p$ is a sum node, we
  have that, since the SPN is complete, the scope of all children of $p$ are the same and are all
  equal to the scope of $p$. Now let $c \in \Ch(p)$. There must exist another child $c$ such that
  $c\neq c^*$ because of lines 5 and 9. From that we have $\Sc(c)=\Sc(c^*)$ because of
  completeness, and since $\Sc(c^*)$ is singular, then $c$ must also be leaf. But it is impossible
  to have leaves with same scope and same parent (line 1 from~\autoref{alg:learngd}). Therefore,
  $p$ is actually a product node.
\end{proof}
%--------------------------------------------------------------------------------------------------

\newpage
\appendix

\newpage

\printbibliography[]

\end{document}
