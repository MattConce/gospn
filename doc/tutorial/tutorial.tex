\documentclass{amsart}

\usepackage[english]{babel}
\usepackage[utf8]{inputenc}
\usepackage{graphicx}
\usepackage{mathtools}
\usepackage{amsthm}
\usepackage{thmtools,thm-restate}
\usepackage{amsfonts}
\usepackage{hyperref}
\usepackage[singlelinecheck=false]{caption}
\usepackage[backend=biber,url=true,doi=true,eprint=false,style=alphabetic]{biblatex}
\usepackage{enumitem}
\usepackage[justification=centering]{caption}
\usepackage{indentfirst}
\usepackage{algorithm}
\usepackage{algpseudocode}
\usepackage{listings}

\addbibresource{references.bib}

\makeatletter
\def\subsection{\@startsection{subsection}{3}%
  \z@{.5\linespacing\@plus.7\linespacing}{.1\linespacing}%
  {\normalfont\itshape}}
\makeatother

\makeatletter
\patchcmd{\@setauthors}{\MakeUppercase}{}{}{}
\makeatother

\DeclareMathOperator*{\argmin}{arg\,min}
\DeclareMathOperator*{\argmax}{arg\,max}

\newcommand\defeq{\mathrel{\overset{\makebox[0pt]{\mbox{\normalfont\tiny\sffamily def}}}{=}}}

\algrenewcommand\algorithmicrequire{\textbf{Input}}
\algrenewcommand\algorithmicensure{\textbf{Output}}

\captionsetup[table]{labelsep=space}

\theoremstyle{plain}

\newcounter{dummy-def}\numberwithin{dummy-def}{section}
\newtheorem{definition}[dummy-def]{Definition}
\newcounter{dummy-thm}\numberwithin{dummy-thm}{section}
\newtheorem{theorem}[dummy-thm]{Theorem}
\newcounter{dummy-prop}\numberwithin{dummy-prop}{section}
\newtheorem{proposition}[dummy-prop]{Proposition}
\newcounter{dummy-ex}\numberwithin{dummy-ex}{section}
\newtheorem{exercise}[dummy-ex]{Exercise}
\newcounter{dummy-eg}\numberwithin{dummy-eg}{section}
\newtheorem{example}[dummy-eg]{Example}

\numberwithin{equation}{section}

\newcommand{\set}[1]{\mathbf{#1}}
\newcommand{\pr}{\mathbb{P}}
\renewcommand{\implies}{\Rightarrow}

\newcommand{\bigo}{\mathcal{O}}

\setlength{\parskip}{1em}

\lstset{frameround=fttt,
	numbers=left,
	breaklines=true,
	keywordstyle=\bfseries,
	basicstyle=\ttfamily,
}

\newcommand{\code}[1]{\lstinline[mathescape=true]{#1}}
\newcommand{\mcode}[1]{\lstinline[mathescape]!#1!}


\title{%
  \noindent\rule{13cm}{1.0pt}\\
  \vspace{0.2cm}
  An Introduction to Sum-Product Networks
  \noindent\rule{13cm}{0.8pt}
}
\xdef\shorttitle{Intro to SPNs}
\author[]{\normalsize\textbf{Renato Lui Geh}\\\small Computer Science\\Institute of Mathematics
  and Statistics\\University of São Paulo\\\texttt{renatolg@ime.usp.br}}

\begin{document}

\begin{abstract}
  Sum-Product Networks (SPNs) are deep probabilistic graphical models (PGMs) that compactly
  represent tractable probability distributions. Exact inference in SPNs is computed in time linear
  in the number of edges, an attractive feature that distinguishes SPNs from other PGMs. However,
  learning SPNs is a tough task. There have been many advances in learning both the structure and
  parameters of SPNs in the past few years. One interesting feature is the fact that we can make
  use of SPN's deep architecture and perform deep learning on these models. Since the number of
  hidden layers not only does not negatively impact the tractability of inference of SPNs but also
  augments the representability of this model, it is very much desirable to continue research on
  deep learning of SPNs. In this article we seek to produce a tutorial on Sum-Product Networks in
  a simpler, clearer way then how it is currently written in literature. We will introduce SPNs
  and explain how knowledge is represented in this model, how to perform exact inference and
  describe in detail a simple structural learning algorithm.
  \vspace*{-3.5em}
\end{abstract}

\maketitle

\section{Introduction}

Hello. I am introducing a very important subject in a clear way.

%--------------------------------------------------------------------------------------------------

\newpage
\appendix

\newpage

\printbibliography[]

\end{document}
